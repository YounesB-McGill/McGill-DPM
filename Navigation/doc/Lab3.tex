\documentclass[twocolumn]{article}

% imports
%\usepackage{times}      % font
\usepackage{graphicx}   % include graphics
\usepackage{fullpage}   % book margins -> std margins
\usepackage{amsmath}    % {align}
\usepackage{wrapfig}    % {wrapfigure}
\usepackage{moreverb}   % \verbatimtabinput
\usepackage[noend,noline]{algorithm2e} % \algorithm
\usepackage{subfig}     % sub-figure
\usepackage{textcomp}   % \textmu
\usepackage{hyperref}   % pdf links
\usepackage{url}        % url support

% def name, id
\def\name{Neil Edelman}
\def\id{110121860}
\def\bname{Alex Bhandari-Young}
\def\bid{260520610}

% ieee style
\bibliographystyle{ieeetr}

% set algoithm comments
\SetKwComment{Comment}{$\bullet$}{}

% define "\fig"
\def\fig#1#2{\begin{figure}[!ht]\begin{center}
\includegraphics[width=0.5\textwidth]{#1.jpg}
\end{center}\caption{#2}\label{#1}\end{figure}}

\def\wide#1#2{\begin{figure*}[hptb]\begin{center}
\includegraphics[height=0.5\textwidth]{#1.jpg}
\end{center}\vspace{-0pt}\caption{#2}\label{#1}\end{figure*}}

% create new commands
\def\^#1{\textsuperscript{#1}}
\def\!{\overline}
\def\degree{\ensuremath{^\circ}}

% lists
\renewcommand{\labelenumi}{\alph{enumi})}
\renewcommand{\labelenumii}{\roman{enumii})}

% for hyperref
\hypersetup{
  colorlinks = true,
  urlcolor = blue,
  linkcolor = blue,
  pdfauthor = {\name},
  pdftitle = {\name -- \id},
  pdfsubject = {A1},
  pdfpagemode = UseNone
}

% info
\author{\bname~--~\bid, \name~--~\id}
\title{Lab 3 -- Navigation (Group~51)}
\date{2013-10-02}

\begin{document}

\maketitle

\abstract{We designed a software system that allows a robot to move to an absolute location while avoiding obstacles.}

%Data Collection (2 sentences + data)
%Observations and Conclusion
%Error Analysis (When possible, specify sub, super, or linear error growth)
%Further Improvements

\section{Data}

Repeated runs of the robot path in Figure~\ref{path} gave the data in Table~\ref{a}. We used more precise measurements values used in the odometer from our last lab\cite{alexneil2}; that made the error go down.

\fig{path}{Robot navigation path in cm.\cite{lab3}}

\begin{table*}[htb]
\begin{center}\begin{tabular}{r@{}l r@{}l r@{}l r@{}l r@{}l r@{}l}
&actual&&& &reported&&& &error&& \\
&x (cm)& &y (cm)& &x (cm)& &y (cm)& &x (cm)& &y (cm) \\
\hline
0&.0& -60&.0& -0&.00& -60&.11& 0&.0& 0&.1 \\
0&.5& -59&.5& 0&.18& -60&.50& 0&.3& 1&.0 \\
0&.5& -59&.5& -0&.10& -60&.50& 0&.6& 1&.0 \\
0&.5& -60&.0& 0&.06& -60&.31& 0&.4& 0&.3 \\
0&.5& -60&.5& 0&.11& -60&.42& 0&.4& -0&.1 \\
0&.0& -59&.0& 0&.17& -60&.51& -0&.2& 1&.5 \\
1&.5& -61&.0& 0&.06& -60&.32& 1&.4& -0&.7 \\
1&.5& -61&.0& -0&.04& -60&.11& 1&.5& -0&.9 \\
1&.0& -60&.0& -0&.09& -59&.99& 1&.1& -0&.0 \\
0&.5& -59&.5& 0&.17& -60&.50& 0&.3& 1&.0 \\
\end{tabular}\end{center}
\caption{Reported error as read by the robot, and real error as read by a ruler and the difference between them.
The difference, as $(x, y)$, mean is $(0.60, 0.33)$, variance is $(0.33, 0.62)$, and the corrected sample standard deviation is $(0.58, 0.79)$.}
\label{a}
\end{table*}

\section{Data Analysis}

\subsection{Discussion}

The errors are part of the odometer. When we measured the odometer values more precisely, our errors dropped significantly. The navigator is just the software that reads from the odometer and decides where to go to minimise the error. It works by comparing a set minimum required error to the actual error and deciding whether that's good enough. We could reduce the error in the software, but that could create a condition where it's never good enough an it loops back incessantly.

\subsection{Error Analysis}

Calculate the differences in Equation~\ref{dx1}--\ref{dy10}.

\begin{align}
d_{x,1} &= (0.0) - (-0.00) \nonumber\\
 &= 0.00 \label{dx1}\\
d_{y,1} &= (-60.0) - (-60.11) \nonumber\\
 &= 0.11 \label{dy1}
\end{align}
\begin{align}
d_{x,2} &= (0.5) - (0.18) \nonumber\\
 &= 0.32 \label{dx2}\\
d_{y,2} &= (-59.5) - (-60.50) \nonumber\\
 &= 1.00 \label{dy2}
\end{align}
\begin{align}
d_{x,3} &= (0.5) - (-0.10) \nonumber\\
 &= 0.60 \label{dx3}\\
d_{y,3} &= (-59.5) - (-60.50) \nonumber\\
 &= 1.00 \label{dy3}
\end{align}
\begin{align}
d_{x,4} &= (0.5) - (0.06) \nonumber\\
 &= 0.44 \label{dx4}\\
d_{y,4} &= (-60.0) - (-60.31) \nonumber\\
 &= 0.31 \label{dy4}
\end{align}
\begin{align}
d_{x,5} &= (0.5) - (0.11) \nonumber\\
 &= 0.39 \label{dx5}\\
d_{y,5} &= (-60.5) - (-60.42) \nonumber\\
 &= -0.08 \label{dy5}
\end{align}
\begin{align}
d_{x,6} &= (0.0) - (0.17) \nonumber\\
 &= -0.17 \label{dx6}\\
d_{y,6} &= (-59.0) - (-60.51) \nonumber\\
 &= 1.51 \label{dy6}
\end{align}
\begin{align}
d_{x,7} &= (1.5) - (0.06) \nonumber\\
 &= 1.44 \label{dx7}\\
d_{y,7} &= (-61.0) - (-60.32) \nonumber\\
 &= -0.68 \label{dy7}
\end{align}
\begin{align}
d_{x,8} &= (1.5) - (-0.04) \nonumber\\
 &= 1.54 \label{dx8}\\
d_{y,8} &= (-61.0) - (-60.11) \nonumber\\
 &= -0.89 \label{dy8}
\end{align}
\begin{align}
d_{x,9} &= (1.0) - (-0.09) \nonumber\\
 &= 1.09 \label{dx9}\\
d_{y,9} &= (-60.0) - (-59.99) \nonumber\\
 &= -0.01 \label{dy9}
\end{align}
\begin{align}
d_{x,10} &= (0.5) - (0.17) \nonumber\\
 &= 0.33 \label{dx10}\\
d_{y,10} &= (-59.5) - (-60.50) \nonumber\\
 &= 1.00 \label{dy10}
\end{align}

Calculate the sum of the differences (Equation~\ref{dx1}--\ref{dy10}) in Equation~\ref{sumx}--\ref{sumy}.

\begin{align}
\text{sum}_{x} &= \sum_{i=1}^{10} d_{x,i} \nonumber\\
 &= (0.00) + \nonumber\\
 &\quad\quad (0.32) + \nonumber\\
 &\quad\quad (0.60) + \nonumber\\
 &\quad\quad (0.44) + \nonumber\\
 &\quad\quad (0.39) + \nonumber\\
 &\quad\quad (-0.17) + \nonumber\\
 &\quad\quad (1.44) + \nonumber\\
 &\quad\quad (1.54) + \nonumber\\
 &\quad\quad (1.09) + \nonumber\\
 &\quad\quad (0.33) \nonumber\\
 &= 5.98 \label{sumx}
\end{align}

\begin{align}
\text{sum}_{y} &= \sum_{i=1}^{10} d_{y,i} \nonumber\\
 &= (0.11) + \nonumber\\
 &\quad\quad (1.00) + \nonumber\\
 &\quad\quad (1.00) + \nonumber\\
 &\quad\quad (0.31) + \nonumber\\
 &\quad\quad (-0.08) + \nonumber\\
 &\quad\quad (1.51) + \nonumber\\
 &\quad\quad (-0.68) + \nonumber\\
 &\quad\quad (-0.89) + \nonumber\\
 &\quad\quad (-0.01) + \nonumber\\
 &\quad\quad (1.00) \nonumber\\
 &= 3.27 \label{sumy}
\end{align}

Calculate the sum of the differences (Equation~\ref{dx1}--\ref{dy10}) squared in Equation~\ref{sum2x}--\ref{sum2y}.

\begin{align}
\text{ssq}_{x} &= \sum_{i=1}^{10} d_{x,i}^{\phantom{x,i}2} \nonumber\\
 &= (0.00)^2 + \nonumber\\
 &\quad\quad (0.32)^2 + \nonumber\\
 &\quad\quad (0.60)^2 + \nonumber\\
 &\quad\quad (0.44)^2 + \nonumber\\
 &\quad\quad (0.39)^2 + \nonumber\\
 &\quad\quad (-0.17)^2 + \nonumber\\
 &\quad\quad (1.44)^2 + \nonumber\\
 &\quad\quad (1.54)^2 + \nonumber\\
 &\quad\quad (1.09)^2 + \nonumber\\
 &\quad\quad (0.33)^2 \nonumber\\
 &= 6.58 \label{sum2x}
\end{align}

\begin{align}
\text{ssq}_{y} &= \sum_{i=1}^{10} d_{y,i}^{\phantom{y,i}2} \nonumber\\
 &= (0.11)^2 + \nonumber\\
 &\quad\quad (1.00)^2 + \nonumber\\
 &\quad\quad (1.00)^2 + \nonumber\\
 &\quad\quad (0.31)^2 + \nonumber\\
 &\quad\quad (-0.08)^2 + \nonumber\\
 &\quad\quad (1.51)^2 + \nonumber\\
 &\quad\quad (-0.68)^2 + \nonumber\\
 &\quad\quad (-0.89)^2 + \nonumber\\
 &\quad\quad (-0.01)^2 + \nonumber\\
 &\quad\quad (1.00)^2 \nonumber\\
 &= 6.65 \label{sum2y}
\end{align}

Calculate the mean from Equation~\ref{sumx}--\ref{sumy} in Equation~\ref{meanx}--\ref{meany}.

\begin{align}
\text{mean}_{x} &= \frac{\text{sum}_{x}}{N} \nonumber\\
 &= \frac{5.98}{10} \nonumber\\
 &= 0.598000 \label{meanx}\\
\text{mean}_{y} &= \frac{\text{sum}_{y}}{N} \nonumber\\
 &= \frac{3.27}{10} \nonumber\\
 &= 0.327000 \label{meany}
\end{align}

Calculate the variance from Equation~\ref{sumx}--\ref{sumy} and \ref{sum2x}--\ref{sum2y} in Equation~\ref{varx}--\ref{vary}.

\begin{align}
\sigma_{x}^{\phantom{x}2} &= \frac{\text{ssq}_{x} - \frac{\text{sum}_{x}^{\phantom{x}2}}{N}}{N-1} \nonumber\\
 &= \frac{6.58 - \frac{5.98^2}{10}}{10-1} \nonumber\\
 &= 0.333684 \label{varx}\\
\sigma_{y}^{\phantom{y}2} &= \frac{\text{ssq}_{y} - \frac{\text{sum}_{y}^{\phantom{y}2}}{N}}{N-1} \nonumber\\
 &= \frac{6.65 - \frac{3.27^2}{10}}{10-1} \nonumber\\
 &= 0.620001 \label{vary}
\end{align}

Calculate the corrected sample standard deviation from the variance (Equation~\ref{varx}--\ref{vary}) in Equation~\ref{stddx}--\ref{stddy}.

\begin{align}
\sigma_x &= \sqrt{\sigma_{x}^{\phantom{x}2}} \nonumber\\
 &= \sqrt{0.333684} \nonumber\\
 &= 0.577654 \label{stddx}\\
\sigma_y &= \sqrt{\sigma_{y}^{\phantom{y}2}} \nonumber\\
 &= \sqrt{0.620001} \nonumber\\
 &= 0.787401 \label{stddy}
\end{align}

\section{Observations and Conclusion}

We noticed experimentally that one actuator is not working, so we placed it on the top of our robot to hold the sensor. As such, the sensor can not move.

We tried {\tt rotate} and {\tt rotateTo} methods as seen in the last lab\cite{alexneil2}, but these inherently blocked output. The {\tt setSpeed} method was more appropriate. {\tt setSpeed} sets it to the absolute value of the speed; you then call {\tt forward} or {\tt backward} depending on the sign. This is not documented in the API.

``In three to four sentences, explain the operation of your controller(s) for navigation. How accurately does it move the robot to its destination? How quickly does it settle (stop oscillating) on its destination? You do not need to provide a quantitative analysis.''

``How would increasing the speed of the robot affect the accuracy of your navigation? What is the main source of error in navigation (and odometry)?''

\section{Further Improvements}

``What steps can be taken to reduce the errors in navigation and odometry? In four (4) to six (6) sentences, Identify at least one hardware and one software solution, and provide an explanation as to why they would work.''

\bibliography{lab3}

\end{document}
