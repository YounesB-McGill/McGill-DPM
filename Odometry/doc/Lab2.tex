\documentclass[twocolumn]{article}

% imports
\usepackage{times}      % font
\usepackage{graphicx}   % include graphics
\usepackage{fullpage}   % book margins -> std margins
\usepackage{amsmath}    % {align}
\usepackage{wrapfig}    % {wrapfigure}
\usepackage{moreverb}   % \verbatimtabinput
\usepackage[noend,noline]{algorithm2e} % \algorithm
\usepackage{subfig}     % sub-figure
\usepackage{textcomp}   % \textmu
\usepackage{hyperref}   % pdf links
\usepackage{url}        % url support

% def name, id
\def\name{Neil Edelman}
\def\id{110121860}
\def\bname{Alex Bhandari-Young}
\def\bid{260520610}

% ieee style
\bibliographystyle{ieeetr}

% set algoithm comments
\SetKwComment{Comment}{$\bullet$}{}

% define "\fig"
\def\fig#1#2{\begin{figure}[!ht]\begin{center}
\includegraphics[width=0.5\textwidth]{#1.jpg}
\end{center}\caption{#2}\label{#1}\end{figure}}

\def\wide#1#2{\begin{figure*}[hptb]\begin{center}
\includegraphics[height=0.5\textwidth]{#1.jpg}
\end{center}\vspace{-0pt}\caption{#2}\label{#1}\end{figure*}}

% create new commands
\def\^#1{\textsuperscript{#1}}
\def\!{\overline}
\def\degree{\ensuremath{^\circ}}

% lists
\renewcommand{\labelenumi}{\alph{enumi})}
\renewcommand{\labelenumii}{\roman{enumii})}

% for hyperref
\hypersetup{
  colorlinks = true,
  urlcolor = blue,
  linkcolor = blue,
  pdfauthor = {\name},
  pdftitle = {\name -- \id},
  pdfsubject = {A1},
  pdfpagemode = UseNone
}

% info
\author{\bname~--~\bid, \name~--~\id}
\title{Lab 2 -- Odometry (Group~51)}
\date{2013-09-25}

\begin{document}

% math eq'ns line up with lists
\abovedisplayskip=-\baselineskip

\maketitle

\abstract{An autonomous automated vehicle is programmed with a dead-reckoning system, and then enhanced with colour sensors aimed at the floor to detect grid lines. The odometer is calibrated and measured.}

%Data Collection (2 sentences + data)
%Observations and Conclusion
%Error Analysis (When possible, specify sub, super, or linear error growth)
%Further Improvements

\section{Data Collection}

Repeated robot runs on the 3-by-3 are seen in without odometer correction in Table~\ref{a} and with odometer correction in Table~\ref{b}.

\begin{table}[htb]
\begin{center}\begin{tabular}{r r@{}l r@{}l r@{}l}
& &x (cm\degree)& &y (cm\degree)& &$\theta (\degree)$ \\
\hline
reported&-107&.13&	92&.29&	0&.20 \\
&-105&.65&	97&.52&	0&.20 \\
&-110&.92&	104&.14&	0&.20 \\
&-150&.94&	108&.81&	0&.20 \\
&27&.33&	224&.08&	0&.20 \\
&24&.80&	92&.78&	0&.20 \\
&16&.90&	-4&.97&	0&.19 \\
&3&.66&	-98&.66&	0&.19 \\
&39&.46&	-48&.36&	0&.19 \\
&52&.39&	4&.79&	0&.19 \\
\hline
total&	-310&.10&	572&.42&	1&.96 \\
(cm)&	-5&.41& 9&.99\\
real (cm)&	31&.2&	18&.3&	24
\end{tabular}\end{center}
\caption{Reported error as read by the robot, and real error as read by a ruler.}
\label{a}
\end{table}

\begin{table}[htb]
\begin{center}\begin{tabular}{r r@{}l r@{}l r@{}l}
& &x (cm\degree)& &y (cm\degree)& &$\theta (\degree)$ \\
\hline
reported&-107&.13&	92&.29&	0&.20 \\
&-105&.65&	97&.52&	0&.20 \\
&-110&.92&	104&.14&	0&.20 \\
&-150&.94&	108&.81&	0&.20 \\
&27&.33&	224&.08&	0&.20 \\
&24&.80&	92&.78&	0&.20 \\
&16&.90&	-4&.97&	0&.19 \\
&3&.66&	-98&.66&	0&.19 \\
&39&.46&	-48&.36&	0&.19 \\
&52&.39&	4&.79&	0&.19 \\
\hline
total&	-310&.10&	572&.42&	1&.96 \\
(cm)&	-5&.41& 9&.99\\
real (cm)&	31&.2&	18&.3&	24
\end{tabular}\end{center}
\caption{With odometry correction. (todo!)}
\label{b}
\end{table}

\section{Data Analysis}

\begin{enumerate}
\item ``What was the standard deviation of the results without correction (compute it for x and y separately, and provide the four (4) values in a table)? Did it decrease when correction was introduced? Explain why/why not.\cite{lab2}'' 4 values? what table?
\item ``With correction, do you expect the error in the x position or the y position to be smaller? Explain.'' The error is reset in the direction orthogonal to travel, so this would have a smaller error.
\end{enumerate}

\section{Observations and Conclusion}

``Is the error you observed in the odometer (without correction) tolerable for larger distances (i.e. circumnavigating the field requires a travel distance five (5) times larger than that used for this lab)? Do you expect the error to grow linearly with respect to travel distance? Explain briefly.\cite{lab2}''

The error grows linearly with dead-reckoning. Without correction, the robot eventually goes astray. With the sensor on the ground, it allows us to get the absolute position every tile. This corrects the odometer and limits the error. It's analogous to the heading gyroscope, which is quick-acting but neutrally-stable, being corrected by the compass, which is slow, but accurate.

\section{Error Analysis}

See Figure~\ref{a} and Figure~\ref{b} for values.

\section{Further Improvements}

\begin{enumerate}
\item ``Propose a means of, in software, reducing the slip of the robot's wheels (do not provide code).\cite{lab2}'' Putting a limit on the speed.
\item ``Propose a means of, in software, correcting the angle reported by the odometer, when (do not provide code):\cite{lab2}''
\begin{enumerate}
\item ``The robot has two light sensors.\cite{lab2}''
\item ``The robot has only one light sensor.\cite{lab2}''
No angle is reported by the odometer; it's just a number representing the forward motion\cite{simpson1989oxford}.
\end{enumerate}
\end{enumerate}

\bibliography{lab2}

\end{document}
