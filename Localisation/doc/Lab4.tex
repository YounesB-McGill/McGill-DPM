\documentclass[twocolumn]{article}

% imports
%\usepackage{times}      % font
\usepackage{graphicx}   % include graphics
\usepackage{fullpage}   % book margins -> std margins
\usepackage{amsmath}    % {align}
\usepackage{wrapfig}    % {wrapfigure}
\usepackage{moreverb}   % \verbatimtabinput
\usepackage[noend,noline]{algorithm2e} % \algorithm
\usepackage{subfig}     % sub-figure
\usepackage{textcomp}   % \textmu
\usepackage{hyperref}   % pdf links
\usepackage{url}        % url support

% def name, id
\def\name{Neil Edelman}
\def\id{110121860}
\def\bname{Alex Bhandari-Young}
\def\bid{260520610}

% ieee style
\bibliographystyle{ieeetr}

% set algoithm comments
\SetKwComment{Comment}{$\bullet$}{}

% define "\fig"
\def\fig#1#2{\begin{figure}[!ht]\begin{center}
\includegraphics[width=0.5\textwidth]{#1.jpg}
\end{center}\caption{#2}\label{#1}\end{figure}}

\def\wide#1#2{\begin{figure*}[hptb]\begin{center}
\includegraphics[height=0.5\textwidth]{#1.jpg}
\end{center}\vspace{-0pt}\caption{#2}\label{#1}\end{figure*}}

% create new commands
\def\^#1{\textsuperscript{#1}}
\def\!{\overline}
\def\degree{\ensuremath{^\circ}}

% lists
\renewcommand{\labelenumi}{\alph{enumi})}
\renewcommand{\labelenumii}{\roman{enumii})}

% for hyperref
\hypersetup{
  colorlinks = true,
  urlcolor = blue,
  linkcolor = blue,
  pdfauthor = {\name},
  pdftitle = {\name -- \id},
  pdfsubject = {A1},
  pdfpagemode = UseNone
}

% info
\author{\bname~--~\bid, \name~--~\id}
\title{Lab 4 -- Localisation (Group~51)}
\date{2013-10-09}

\begin{document}

\maketitle

\abstract{Abstract.\cite{lab4}}

%Data Collection (2 sentences + data)
%Observations and Conclusion
%Error Analysis (When possible, specify sub, super, or linear error growth)
%Further Improvements

\section{Data}

\cite{alexneil3}

%\fig{path}{Robot navigation path in cm.\cite{lab3}}

\begin{table*}[htb]
\begin{center}\begin{tabular}{r@{}l r@{}l r@{}l r@{}l r@{}l r@{}l}
&actual&&& &reported&&& &error&& \\
&x (cm)& &y (cm)& &x (cm)& &y (cm)& &x (cm)& &y (cm) \\
\hline
0&.0& -60&.0& -0&.00& -60&.11& 0&.0& 0&.1 \\
0&.5& -59&.5& 0&.18& -60&.50& 0&.3& 1&.0 \\
0&.5& -59&.5& -0&.10& -60&.50& 0&.6& 1&.0 \\
0&.5& -60&.0& 0&.06& -60&.31& 0&.4& 0&.3 \\
0&.5& -60&.5& 0&.11& -60&.42& 0&.4& -0&.1 \\
0&.0& -59&.0& 0&.17& -60&.51& -0&.2& 1&.5 \\
1&.5& -61&.0& 0&.06& -60&.32& 1&.4& -0&.7 \\
1&.5& -61&.0& -0&.04& -60&.11& 1&.5& -0&.9 \\
1&.0& -60&.0& -0&.09& -59&.99& 1&.1& -0&.0 \\
0&.5& -59&.5& 0&.17& -60&.50& 0&.3& 1&.0 \\
\end{tabular}\end{center}
\caption{$(0.58, 0.79)$ this is the last lab . . .}
\label{a}
\end{table*}

\section{Data Analysis}

\subsection{Discussion}

\subsection{Error Analysis}

Automatic.

\section{Observations and Conclusion}

We attached our defective servomotor to the top of our robot; we then attached menacing parts to it, and set it spinning at top speed. This added personality.

\section{Further Improvements}

\bibliography{lab4}

\end{document}
