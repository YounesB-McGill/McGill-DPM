\documentclass[twocolumn]{article}

% imports
%\usepackage{times}      % font
\usepackage{graphicx}   % include graphics
\usepackage{fullpage}   % book margins -> std margins
\usepackage{amsmath}    % {align}
\usepackage{wrapfig}    % {wrapfigure}
\usepackage{moreverb}   % \verbatimtabinput
\usepackage[noend,noline]{algorithm2e} % \algorithm
\usepackage{subfig}     % sub-figure
\usepackage{textcomp}   % \textmu
\usepackage{hyperref}   % pdf links
\usepackage{url}        % url support

% def name, id
\def\name{Neil Edelman}
\def\id{110121860}
\def\bname{Alex Bhandari-Young}
\def\bid{260520610}

% ieee style
\bibliographystyle{ieeetr}

% set algoithm comments
\SetKwComment{Comment}{$\bullet$}{}

% define "\fig"
\def\fig#1#2{\begin{figure}[!ht]\begin{center}
\includegraphics[width=0.5\textwidth]{#1.jpg}
\end{center}\caption{#2}\label{#1}\end{figure}}

\def\wide#1#2{\begin{figure*}[hptb]\begin{center}
\includegraphics[height=0.5\textwidth]{#1.jpg}
\end{center}\vspace{-0pt}\caption{#2}\label{#1}\end{figure*}}

% create new commands
\def\^#1{\textsuperscript{#1}}
\def\!{\overline}
\def\degree{\ensuremath{^\circ}}

% lists
\renewcommand{\labelenumi}{\arabic{enumi}.}
\renewcommand{\labelenumii}{\roman{enumii})}

% for hyperref
\hypersetup{
  colorlinks = true,
  urlcolor = blue,
  linkcolor = blue,
  pdfauthor = {\name},
  pdftitle = {\name -- \id},
  pdfsubject = {A1},
  pdfpagemode = UseNone
}

% info
\author{\bname~--~\bid, \name~--~\id}
\title{Lab 5 -- Searching for Objects (Group~51)}
\date{2013-10-20}

\begin{document}

\maketitle

\abstract{This lab has two parts. The first is recognizing different types of objects using the color sensor. The second is navigating the field while searching for blocks. These are combined with localization from the previous lab. First the robot localises in a set environment. It then searches for the foam object in a sea of wooden blocks, captures, and brings it to the set destination location in the allotted five minutes.}

%Data Collection (2 sentences + data)
%Observations and Conclusion
%Error Analysis (When possible, specify sub, super, or linear error growth)
%Further Improvements

\section{Data and Analysis}

\subsection{Calibrating}

The experimental calibration data are found in Table~\ref{styrofoam} and Table~\ref{wood}. The means were the values used in the software to distinguish wood from styrofoam.

\begin{table}[htb]
\centering
\begin{tabular}{@{(}l@{, }l@{, }l@{)} l @{(}l@{, }l@{, }l@{)}}
R & G & B & $\rightarrow$ & R & G & B \\
\hline
21 & 16 & 21 &  & 0.62 & 0.47 & 0.62 \\
57 & 50 & 61 &  & 0.59 & 0.51 & 0.63 \\
176 & 187 & 188 &  & 0.55 & 0.59 & 0.59 \\
174 & 185 & 186 &  & 0.55 & 0.59 & 0.59 \\
68 & 63 & 76 &  & 0.57 & 0.53 & 0.63 \\
87 & 84 & 99 &  & 0.56 & 0.54 & 0.63 \\
208 & 227 & 219 &  & 0.55 & 0.60 & 0.58 \\
\end{tabular}
\caption{Calibrating the colour of the styrofoam. The second column is the normalisiation of the raw data.
The mean is $(0.57, 0.55, 0.61)$, variance is $(0.0007, 0.0022, 0.0005)$, and the corrected sample standard deviation is $(0.026, 0.047, 0.023)$.}
\label{styrofoam}
\end{table}

\begin{table}[htb]
\centering
\begin{tabular}{@{(}l@{, }l@{, }l@{)} l @{(}l@{, }l@{, }l@{)}}
R & G & B & $\rightarrow$ & R & G & B \\
\hline
118 & 66 & 64 &  & 0.79 & 0.44 & 0.43 \\
119 & 67 & 65 &  & 0.79 & 0.44 & 0.43 \\
106 & 53 & 50 &  & 0.82 & 0.41 & 0.39 \\
152 & 103 & 98 &  & 0.73 & 0.49 & 0.47 \\
97 & 48 & 49 &  & 0.82 & 0.40 & 0.41 \\
143 & 80 & 68 &  & 0.81 & 0.45 & 0.38 \\
125 & 65 & 59 &  & 0.82 & 0.43 & 0.39 \\
\end{tabular}
\caption{Calibrating the colour of the wood. The second column is the normalisiation of the raw data.
The mean is $(0.80, 0.44, 0.41)$, variance is $(0.0010, 0.0009, 0.0010)$, and the corrected sample standard deviation is $(0.032, 0.030, 0.032)$.}
\label{wood}
\end{table}

\begin{enumerate}

\subsection{Object detection}

\item
``Perform at least 10 trials of object recognition using an object other than the Styrofoam block and note the number of false positives.\cite{lab5}''

The other-object data is in Table~\ref{qwood}. We tested the colour object recognition on wooden blocks (denoted "block" in the table) as well as on the floor and walls of the arena and empty space. The "recognizes" column refers to the assessment made by the algorithm, which reports whether the object in question is more likely wooden or styrofoam. The percentage calculated in the percent column refers to the certainty of the algorithm's assesment. It is apparent that the sensor is very effective, especially at close distances. The arena is made entirely of wood, and we can successfully recognize it as well as the blocks as wooden. There were no false positives when testing the wooden blocks as well as when testing the arena itself, as can be seen from the table. In open space the algorithm is designed to still detect wooden or styrofoam so there are two false positives. However, the sensor is only used when a block is within range as determined by the ultrasonic sensor so this should not be a problem. Additionally, values with low certainty (close to 50 percent) can be filtered out.

\begin{table}[htb]\footnotesize
\begin{center}\begin{tabular}{llll}
object& cm& recognises& percent \\
\hline
space&   & wood& 50\% \\
space&   & wood& 50\% \\
space&   & equal& \\
space&   & equal& 50\% \\
\hline
block& 50& wood& 51\% \\
block& 45& wood& 53\% \\
block& 40& wood& 57\% \\
block& 35& wood& 68\% \\
block& 30& wood& 74\% \\
block& 25& wood& 82\% \\
block& 20& wood& 81\% \\
block& 15& wood& 88\% \\
block& 10& wood& 95\% \\
block&  5& wood& 98\% \\
floor& 50& wood& 52\% \\
floor& 20& wood& 84\% \\
floor& 10& wood& 94\% \\
side& 50& wood& 76\% \\
side& 20& wood& 84\% \\
side& 10& wood& 94\% \\
side&  5& wood& 99\% \\
\end{tabular}\end{center}
\caption{Space is empty space giving $(0,0,0)$ or very close;
it has not been included. The differences can be explained by the orientation.}
\label{qwood}
\end{table}

\item
``Repeat the above step, but using the Styrofoam block each time, noting the total number of false negatives. Ideally both errors should be as small as possible.\cite{lab5}''

The styrofoam data is is Table~\ref{qstyrofoam}. We used an even greater range of distances when testing on the styrofoam blocks. From the data we can conclude that our algorithm is reliably able to differentiate the styrofoam from about 30\,cm. It starts to recoginze at distances less than 70cm. Again at close distances the sensor is extremely accurate. There are no false positives at distances less than 70cm. And within the required operating range of this sensor, which is around 30cm and below, the certainty is high. 

\begin{table*}[htb]\footnotesize
\begin{center}\begin{tabular}{llll}
object& cm& recognises& percent \\
\hline
styrofoam& 90& wood& 50\% \\
styrofoam& 90& styrofoam& 50\% \\
styrofoam& 90& wood& 50\% \\
styrofoam& 90& styrofoam& 50\% \\
styrofoam& 90& wood& 50\% \\
styrofoam& 90& wood& 50\% \\
styrofoam& 80& styrofoam& 50\% \\
styrofoam& 70& styrofoam& 51\% \\
styrofoam& 60& styrofoam& 54\% \\
styrofoam& 50& styrofoam& 53\% \\
styrofoam& 40& styrofoam& 59\% \\
styrofoam& 30& styrofoam& 71\% \\
styrofoam& 20& styrofoam& 84\% \\
styrofoam& 10& styrofoam& 97\% \\
styrofoam&  5& styrofoam& 98\% \\
\end{tabular}\end{center}
\caption{Distance that the colour sensor can differentiate a styrofoam block.}
\label{qstyrofoam}
\end{table*}

\subsection{Searching for Objects}

\item
``Run through the search program at least 5 times, recording the average time taken to localize, to find a block, and then to travel to the destination. In the final competition, you must complete localization in less than 30 seconds. Also, estimate your localization and final destination errors for each trial.\cite{lab5}''

Table~\ref{open} is from Lab 4\cite{alexneil4}. It gives the error incurred from in-place localisation. The robot was placed in the first tile facing different angles and localized. The error in theta is calculated. Table~\ref{robot-wars} gives the time for each step (localising, finding the block, traveling to the destination) and the total time taken. It also gives an estimate on the error when reaching the end; this error is not normally distributed, it is a function of how many collisions the robot took.

\begin{table*}[htb]
\begin{center}\begin{tabular}{r@{}l r@{}l r@{}l r@{}l}
&$\theta_{\text{start}}$ (\degree)& &$\theta_{\text{final}}$ (\degree)& &$\theta_{\text{reported}}$ (\degree)& &$\theta_{\text{error}}$ (cm) \\
\hline
45&& 359&& 1&.8366& -2&.8366 \\
0&& 359&& 0&.9664& -1&.9664 \\
315&& 0&& 1&.6627& -1&.6627 \\
270&& 0&& 1&.8364& -1&.8364 \\
225&& 359&& 1&.4885& -2&.4885 \\
180&& 359&& 1&.9662& -2&.9662 \\
135&& 0&& 1&.8364& -1&.8364 \\
90&& 359&& 1&.0112& -2&.0112 \\
345&& 0&& 1&.0104& -1&.0104 \\
15&& 359&& 2&.1846& -3&.1846 \\
\end{tabular}\end{center}
\caption{The error mean is $-2.18$\,cm, variance is %$0.4555$
$0.46$\,cm$^{2}$, and the corrected sample standard deviation is %$0.6749$
$0.67$\,cm.\cite{alexneil4}}
\label{open}
\end{table*}

\begin{table*}[htb]
\begin{center}\begin{tabular}{l l l l l}
localising (s) & finding (s) & travelling (s) & total (s) & error (cm)\\
\hline
15& 180& 4& 199& 30 \\
16& 220& 4& 240& 5 \\
15& 360& 7& 382& 40 \\
14& 120& 6& 140& 1 \\
15& 320& 7& 342& 3 \\
16& 290& 4& 310& 30 \\
\end{tabular}\end{center}
\caption{The error mean is $18$\,cm, variance is $291$\,cm$^{2}$, and the corrected sample standard deviation is $17$\,cm.}
\label{robot-wars}
\end{table*}

\end{enumerate}

\clearpage

\section{Error Calculations}

The localisation error in Table~\ref{open}.\cite{alexneil4}

Calculate the differences in Equation~\ref{open-d1}--\ref{open-d10}.

\begin{align}
d_{1} &= ((359) - (1.8366))_{360[-180,180]} \nonumber\\
 &= -2.8366 \label{open-d1}
\end{align}
\begin{align}
d_{2} &= ((359) - (0.9664))_{360[-180,180]} \nonumber\\
 &= -1.9664 \label{open-d2}
\end{align}
\begin{align}
d_{3} &= ((0) - (1.6627))_{360[-180,180]} \nonumber\\
 &= -1.6627 \label{open-d3}
\end{align}
\begin{align}
d_{4} &= ((0) - (1.8364))_{360[-180,180]} \nonumber\\
 &= -1.8364 \label{open-d4}
\end{align}
\begin{align}
d_{5} &= ((359) - (1.4885))_{360[-180,180]} \nonumber\\
 &= -2.4885 \label{open-d5}
\end{align}
\begin{align}
d_{6} &= ((359) - (1.9662))_{360[-180,180]} \nonumber\\
 &= -2.9662 \label{open-d6}
\end{align}
\begin{align}
d_{7} &= ((0) - (1.8364))_{360[-180,180]} \nonumber\\
 &= -1.8364 \label{open-d7}
\end{align}
\begin{align}
d_{8} &= ((359) - (1.0112))_{360[-180,180]} \nonumber\\
 &= -2.0112 \label{open-d8}
\end{align}
\begin{align}
d_{9} &= ((0) - (1.0104))_{360[-180,180]} \nonumber\\
 &= -1.0104 \label{open-d9}
\end{align}
\begin{align}
d_{10} &= ((359) - (2.1846))_{360[-180,180]} \nonumber\\
 &= -3.1846 \label{open-d10}
\end{align}

Calculate the sum of the differences (Equation~\ref{open-d1}--\ref{open-d10}) in Equation~\ref{open-sum}.

\begin{align}
\text{sum} &= \sum_{i=1}^{10} d_{i} \nonumber\\
 &= (-2.8366) + \nonumber\\
 &\quad\quad (-1.9664) + \nonumber\\
 &\quad\quad (-1.6627) + \nonumber\\
 &\quad\quad (-1.8364) + \nonumber\\
 &\quad\quad (-2.4885) + \nonumber\\
 &\quad\quad (-2.9662) + \nonumber\\
 &\quad\quad (-1.8364) + \nonumber\\
 &\quad\quad (-2.0112) + \nonumber\\
 &\quad\quad (-1.0104) + \nonumber\\
 &\quad\quad (-3.1846) \nonumber\\
 &= -21.7994 \label{open-sum}
\end{align}

Calculate the sum of the differences (Equation~\ref{open-d1}--\ref{open-d10}) squared in Equation~\ref{open-sum2}.

\begin{align}
\text{ssq} &= \sum_{i=1}^{10} d_{i}^{\phantom{i}2} \nonumber\\
 &= (-2.84)^2 + \nonumber\\
 &\quad\quad (-1.97)^2 + \nonumber\\
 &\quad\quad (-1.66)^2 + \nonumber\\
 &\quad\quad (-1.84)^2 + \nonumber\\
 &\quad\quad (-2.49)^2 + \nonumber\\
 &\quad\quad (-2.97)^2 + \nonumber\\
 &\quad\quad (-1.84)^2 + \nonumber\\
 &\quad\quad (-2.01)^2 + \nonumber\\
 &\quad\quad (-1.01)^2 + \nonumber\\
 &\quad\quad (-3.18)^2 \nonumber\\
 &= 51.62 \label{open-sum2}
\end{align}

Calculate the mean from Equation~\ref{open-sum} in Equation~\ref{open-mean}.

\begin{align}
\text{mean} &= \frac{\text{sum}}{N} \nonumber\\
 &= \frac{-21.7994}{10} \nonumber\\
 &= -2.179940 \label{open-mean}
\end{align}

Calculate the variance from Equation~\ref{open-sum} and \ref{open-sum2} in Equation~\ref{open-var}.

\begin{align}
\sigma^{2} &= \frac{\text{ssq} - \frac{\text{sum}^{2}}{N}}{N-1} \nonumber\\
 &= \frac{51.6208 - \frac{-21.7994^2}{10}}{10-1} \nonumber\\
 &= 0.455492 \label{open-var}
\end{align}

Calculate the corrected sample standard deviation from the variance (Equation~\ref{open-var}) in Equation~\ref{open-stdd}.

\begin{align}
\sigma &= \sqrt{\sigma^{2}} \nonumber\\
 &= \sqrt{0.455492} \nonumber\\
 &= 0.674902 \label{open-stdd}
\end{align}






This is the error on Table~\ref{robot-wars}.

Calculate the sum of the error in Equation~\ref{robot-wars-sum}.

\begin{align}
\text{sum} &= \sum_{i=1}^{6} d_{i} \nonumber\\
 &= (30) + \nonumber\\
 &\quad\quad (5) + \nonumber\\
 &\quad\quad (40) + \nonumber\\
 &\quad\quad (1) + \nonumber\\
 &\quad\quad (3) + \nonumber\\
 &\quad\quad (30) \nonumber\\
 &= 109 \label{robot-wars-sum}
\end{align}

Calculate the sum of the error squared in Equation~\ref{robot-wars-sum2}.

\begin{align}
\text{ssq} &= \sum_{i=1}^{6} d_{i}^{\phantom{i}2} \nonumber\\
 &= (30.00)^2 + \nonumber\\
 &\quad\quad (5.00)^2 + \nonumber\\
 &\quad\quad (40.00)^2 + \nonumber\\
 &\quad\quad (1.00)^2 + \nonumber\\
 &\quad\quad (3.00)^2 + \nonumber\\
 &\quad\quad (30.00)^2 \nonumber\\
 &= 3435 \label{robot-wars-sum2}
\end{align}

Calculate the mean from Equation~\ref{robot-wars-sum} in Equation~\ref{robot-wars-mean}.

\begin{align}
\text{mean} &= \frac{\text{sum}}{N} \nonumber\\
 &= \frac{109}{6} \nonumber\\
 &= 18.166667 \label{robot-wars-mean}
\end{align}

Calculate the variance from Equation~\ref{robot-wars-sum} and \ref{robot-wars-sum2} in Equation~\ref{robot-wars-var}.

\begin{align}
\sigma^{2} &= \frac{\text{ssq} - \frac{\text{sum}^{2}}{N}}{N-1} \nonumber\\
 &= \frac{3435.0000 - \frac{109.0000^2}{6}}{6-1} \nonumber\\
 &= 290.966667 \label{robot-wars-var}
\end{align}

Calculate the corrected sample standard deviation from the variance (Equation~\ref{robot-wars-var}) in Equation~\ref{robot-wars-stdd}.

\begin{align}
\sigma &= \sqrt{\sigma^{2}} \nonumber\\
 &= \sqrt{290.966667} \nonumber\\
 &= 17.057745 \label{robot-wars-stdd}
\end{align}

\section{Observations and Conclusions}

We did a literature search and discovered that some variant of Kalman filter is used almost exclusively for robot navigation; for example in \cite{davison2003real}. We did not use this because of time constraints, our robot following a heuristic path. We implemented a PID-control class. We have a light sensor and a colour sensor, so we could implement odometer correction; however, we wanted our robot to go at all angles anywhere in the field. Odometer correction was not tested with this relaxed constraint.

\begin{enumerate}

\item ``What differences, if any, did you observe in the behavior/performance of your earlier code (i.e. localization, odometry, navigation) when combined in a larger system? Explain any discrepancies. If it turns out that things worked out pretty much as expected, explain how the design of your code contributed.''

We spent a large amout of time on the structure of our code in previous labs, and sorted out a lot of problems with integrating the odometry and navigation with localisation in Lab 4. Because of this, when we implemented these parts into this lab, things when rather smoothly. However, we were still not happy with the structure of the code, and started on a redesign which is much more intuitive, but we could not get it working reliably in time for this lab so we used the old code from lab 4 that we know worked. It terms of the structure of this code, we solved a lot of our problems from previous labs by abstracting all of the required methods from labs 2 and 3 into one class, the odometer, which contains navigation. Calling odometer.travelTo() calls the travelTo method in navigation. This made using the code much simpler, but it is still not a very intuitive solution. In the redesign, which we hope to use in the final project, we would like everything contained in a robot controller class, making the code much cleaner.

\item ``How reliable was your object detection? What factors influence the reliability of object detection? Where would you expect your code to break down? What steps can you take to make detection more robust?''

The object detection was very solid. We spent a lot of time making it robust and independant of environmental factors such as lighting. We used a method that is $\mathcal{O}(2^{n-1})$, but n is constant and two (styrofoam and wood.) We normalise our colours to make it lighting-independent. We compare with experimental values for the different substances, and pick the closest Cartesian distance to normalised colour values. 

The normalisation projects the HSL values onto L = 0.5 so it isn't affected by natural light. The barycentric coordinates are the square of these values. We could have multiplied the eigenvalues by their relative sensitivities, but we didn't know them; the blue was suspected to be lower when we tested it.

The comparison introduces an error since we are working in normalised spherical co-ordinates. It's monotonic, which is all we care about when comparing. The approximate error is maximally seen in Equation~\ref{eq:cool}.

\begin{align}
\int_{x=0}^{\sqrt{2}} \left(
	\sqrt{1-y^{2}} - (1 - y)
\right) dy &\approx 0.19
\label{eq:cool}
\end{align}

We could try turning the flood light on while determining colour.

While there are not many situations where the code would break down within the limitations of Lab 5 (short of very drastic changes in lighting, which were not tested), in the final project the introduction of an unexpected object, namely the opposing teams robot, could effect the algorithm. This could be accounted for by testing the sensor on the expected colors of the NXT robot. This should not be a problem as the sensor looks for blue or wood colors, both of which are unlike the colors of the NXT. Unless the opposing team purposely uses blue or wood colors to confuse our robot our code should function properly. Additional thought could be put into dealing with a blue robot, but this seems very unlikely.

\item ``What aspect of this lab did you find most difficult? What aspect of this lab did you find most surprising or unexpected?''

We had the most trouble in trying to redesign our current code structure to make it easier to use. In transferring the code of lab 1--4\cite{alexneil1,alexneil2,alexneil3,alexneil4} to the new design a lot of stuff just didn't work. We decided to use a different coordinate system from that provided in the previous lab\cite{iso}. Because the odometer and navigation were designed to use this system, we had some difficulty and were not able to complete the redesign, and ended up reverting to the old design used in lab 4. This had already been tested, and was more robust. We would like to used the redesigned code because it is more intuitive and thus easier to work with and much smaller code size. The mechanical planning of the robot was fun.

\end{enumerate}

\section{Further Improvements}

The colour detection worked very well. Also the front bumper on the robot designed as a capture system for styrofoam blocks was simple and worked well. These systems could be refined, but we are confident the overall design is good enough to be used in the competition. As mentioned before, we would like a very robust and intuitive robot class that consolidates all of the previous lab code. This will be extremely important when the code gets more complex for the final project. The current version of this code redesign was buggy. However, this is expected to happen when multiple sources are merged. Hopefully with a little more work we can get it reliable.

\bibliography{lab5}

\end{document}
